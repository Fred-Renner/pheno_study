\subsection*{N\+Tuples}

This software works with ntuples produced using \href{https://gitlab.cern.ch/hh4b/XhhCommon/}{\tt Xhh\+Common}. Suitable N\+Tuples are available on the grid\+:


\begin{DoxyItemize}
\item {\ttfamily user.\+saparede.\+H\+H4\+B.\+2018-\/04-\/29\+T1237\+Z.\+data17-\/rel21} {\bfseries B\+E\+W\+A\+RE -- N\+OT B\+L\+I\+N\+D\+ED. DO N\+OT L\+O\+OK AT S\+I\+G\+N\+AL R\+E\+G\+I\+ON.}
\item {\ttfamily user.\+bstanisl.\+H\+H4\+B.\+2018-\/04-\/30\+T0908\+Z.\+data16-\/rel21}
\item {\ttfamily user.\+bstanisl.\+H\+H4\+B.\+2018-\/04-\/30\+T0908\+Z.\+data15-\/rel21}
\end{DoxyItemize}

However, the 2015 and 2016 N\+Tuples contain a few corrupt files, so the reconstruction needs to be run one file at a time.

\subsection*{Running the Reconstruction}

The reconstruction software requires R\+O\+OT 6.\+12, and the kinematic reweighting and plotting scripts also require Python 3.\+6. The easiest way to get these on L\+X\+P\+L\+US or a Tier 3 is to use the L\+CG 93python3 release with\+:


\begin{DoxyCode}
$ source /cvmfs/sft.cern.ch/lcg/views/LCG\_93python3/x86\_64-slc6-gcc7-opt/setup.sh
\end{DoxyCode}


The code can then be built with C\+Make\+:


\begin{DoxyCode}
$ mkdir build
$ cd build
$ cmake ..
$ make
\end{DoxyCode}


The executable will then be found at {\ttfamily build/resolved-\/recon}. The help text is\+:


\begin{DoxyCode}
$ ./resolved-recon --help
Reconstruct and filter events in the HH->4b resolved channel
Usage:
  resolved-recon [OPTION...] input files

 main options:
  -f, arg              Untagged to tagged normalization (default: 0.22)
  -m, --mc\_config arg  Config file if MC
      --tree arg       Tree name (default: XhhMiniNtuple)
      --top\_veto       Top Veto (default: false)
      --grl arg        GRL
  -o, --output arg     Output file (default: reconstructed.root)
  -h, --help           Help
\end{DoxyCode}


It can be run on data with a command like\+:


\begin{DoxyCode}
$ ~/hh4b/recon/build/resolved-recon -f 0.16 --top\_veto --grl
       ~/hh4b/GRLs/data16\_13TeV.periodAllYear\_DetStatus-v89-pro21-01\_DQDefects-00-02-04\_PHYS\_StandardGRL\_All\_Good\_25ns\_BjetHLT.xml -o data16.root
       /data/atlas/atlasdata/hh4b-ntuples/**/*.root
\end{DoxyCode}


In particular, notice the {\ttfamily -\/-\/grl} option which takes the G\+RL xml file, and the {\ttfamily -\/-\/top\+\_\+veto} option to enable the top veto. You should correct the paths to the executable, G\+RL, and input N\+Tuples to match your setup.

As you can see, the reconstruction can be run on many input files at once. However, if some of these files are corrupt, they can cause a segfault (this is fixed in the upcoming \href{https://root-forum.cern.ch/t/tdataframe-segfault-on-error-reading-file/29081}{\tt R\+O\+OT 6.\+14}, and this reconstruction software will be updated for this R\+O\+OT version once it is in an L\+CG release). As a result, on data it\textquotesingle{}s best to run the reconstruction on one or a few files at a time. {\ttfamily find} and {\ttfamily xargs} can be helpful for this.

\subsubsection*{Running on MC}

When running on MC, the reconstruction must be run on all files at once, to correctly calculate the event weights. In addition, the {\ttfamily -\/-\/grl} option should not be given, but instead the {\ttfamily -\/-\/mc\+\_\+config} option should be used, with its argument being a J\+S\+ON file describing the Monte-\/\+Carlo configuration. An example configuration file is in {\ttfamily ttbar.\+nonallhad.\+json}. The {\ttfamily lumi} variable in this file is the total integrated luminosity the MC should be normalized to (3.\+1 for 2015 data only).

\subsection*{Kinematic Reweighting and Plotting}

The code for kinematic reweighting and plotting can be found in the {\ttfamily tools} directory. In addition to the L\+CG release, these require the \href{https://github.com/scikit-hep/root_pandas}{\tt root\+\_\+pandas} and \href{https://github.com/beojan/atlas-mpl}{\tt atlas\+\_\+mpl\+\_\+style} Python modules, both of which can be installed from P\+IP\+:


\begin{DoxyCode}
$ pip install root\_pandas
$ pip install atlas-mpl-style
\end{DoxyCode}


Having done this, copy {\ttfamily kinematic-\/reweighting.\+py} and {\ttfamily plot.\+py} into the same directory as the output from the reweighting software. Then, update the {\ttfamily input\+\_\+file} in {\ttfamily kinematic-\/reweighting.\+py} and run


\begin{DoxyCode}
$ ./kinematic-reweighting.py
\end{DoxyCode}


This will produce {\ttfamily Iterations.\+pdf} which shows the outcome of each iteration of the reweighting, and {\ttfamily reweight.\+pickle} which contains the reweighting information in a format the plotting tool can understand. It will also print an optimized normalization and n-\/jets factor, which should be passed to {\ttfamily plot.\+py}. The plotting tool is then run with a command like


\begin{DoxyCode}
$ ./plot.py --norm 0.10031 -f 0.18747 signal m\_hh data16.root
\end{DoxyCode}


The help text for this is\+:


\begin{DoxyCode}
$ ./plot.py --help
usage: plot.py [-h] [--no-kinematic-reweighting] [--mc MC\_label MC\_file]
               [--norm NORM] [-f F]
               \{signal,control,sideband\}
               \{m\_hh,m\_h1,pT\_h1,eta\_h1,m\_h2,pT\_h2,eta\_h2,njets,pT\_4,pT\_2,eta\_i,dRjj\_1,dRjj\_2\}
               data\_file

Plot data and background (QCD and MC)

positional arguments:
  \{signal,control,sideband\}
                        HH mass region
  \{m\_hh,m\_h1,pT\_h1,eta\_h1,m\_h2,pT\_h2,eta\_h2,njets,pT\_4,pT\_2,eta\_i,dRjj\_1,dRjj\_2\}
                        Variable to plot
  data\_file             Data ROOT file

optional arguments:
  -h, --help            show this help message and exit
  --no-kinematic-reweighting
  --mc MC\_label MC\_file
                        MC backgrounds
  --norm NORM           QCD normalization
  -f F                  n-jets factor
\end{DoxyCode}
 